\documentclass[11pt,a4paper]{article}

% Essential packages
\usepackage[utf8]{inputenc}
\usepackage[T1]{fontenc}
\usepackage{amsmath,amsthm,amssymb}
\usepackage{mathtools}
\usepackage{graphicx}
\usepackage[colorlinks=true,linkcolor=blue,citecolor=blue]{hyperref}
\usepackage{algorithm}
\usepackage{algpseudocode}
\usepackage{tikz}

% Page layout
\usepackage[top=2.5cm,bottom=2.5cm,left=2.5cm,right=2.5cm]{geometry}

% Custom theorem environments
\theoremstyle{plain}
\newtheorem{theorem}{Theorem}[section]
\newtheorem{lemma}[theorem]{Lemma}
\newtheorem{proposition}[theorem]{Proposition}
\newtheorem{corollary}[theorem]{Corollary}

\theoremstyle{definition}
\newtheorem{definition}[theorem]{Definition}
\newtheorem{example}[theorem]{Example}

\theoremstyle{remark}
\newtheorem{remark}[theorem]{Remark}

% Document info
\title{Notes on Loop-Erased Random Walk Simulations}
\author{Me}
\date{\today}

% Custom commands for LERW notation
\newcommand{\lerw}{\text{LERW}}
\newcommand{\prob}{\mathbb{P}}
\newcommand{\expect}{\mathbb{E}}
\newcommand{\Z}{\mathbb{Z}}

\begin{document}
\maketitle

\begin{abstract}
These notes discuss simulation techniques for Loop-Erased Random Walks (LERW), including implementation details, algorithmic considerations, and analysis of results.
\end{abstract}

\section{Introduction}

\section{Theoretical Background}
\subsection{Definition of LERW}

\section{Simulation Algorithm}
\begin{algorithm}
\caption{Loop-Erased Random Walk Generation}
\begin{algorithmic}[1]
\end{algorithmic}
\end{algorithm}

\section{Implementation Details}

\section{Results and Analysis}

\section{Conclusions}

\bibliographystyle{plain}
\bibliography{references}

\end{document}